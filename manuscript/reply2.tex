\documentclass[12pt]{article}
\usepackage[margin=1in]{geometry}
\usepackage{natbib}
\usepackage{hyperref}

\usepackage{listings}
\usepackage{rotating, graphicx}
\graphicspath{{./}, {./image/}}
\usepackage{booktabs, natbib}
% \usepackage{breakurl}
% \usepackage [english]{babel}
\usepackage{amsmath, amsbsy, amsthm, epsfig, epsf, psfrag, graphicx, 
amssymb, enumerate}
\usepackage{bm}
\usepackage{multirow, multicol}

\usepackage[dvipsnames]{color}
\definecolor{darkblue}{rgb}{0.1, 0.2, 0.6}

\newcommand{\jy}[1]{\textcolor{red}{JY: #1}}
\newcommand{\eds}[1]{\textcolor{blue}{(EDS: #1)}}
\newcommand{\mc}[1]{\textcolor{green}{(MC: #1)}}
\newcommand{\xz}[1]{\textcolor{cyan}{(XZ: #1)}}

\sloppy

% \usepackage{csquotes}
% \usepackage [autostyle, english = american]{csquotes}
% \MakeOuterQuote{"}

% \usepackage{bibentry}
\newenvironment{comment}%
{\begin{quotation}\noindent\small\it\color{darkblue}\ignorespaces%
}{\end{quotation}}


\begin{document}

\begin{center}
  {\Large\bf Response to the Comments}
\end{center}


We thank the Editor and Associate Editor for the opportunity to add further detail to our manuscript.
We have taken the revision process seriously in order
to meet the expectations for final acceptance.


The revised manuscript includes one notable change:
\begin{enumerate}
\item A supplement has been added, which includes both
our previous exploration of automatic block size selection as
well as our new simulations for block bootstrap under
\citet{babu2004goodness}'s bias correction.
\end{enumerate}


Point-by-point responses to the comments are as follows, with the
comments quoted in \emph{\color{darkblue} italic and blue}.


\subsection*{To AE}

\begin{comment}
The paper is improved from the first submission.


I would still like a little bit of clarification about the bias
correction $K_n(x)$ versus the bias correction $C_n(x)$. I understand
that $K_n(x)$, which the authors prescribe under the circular blocks
bootstrap, is based on bootstrap expectations $E_*$ of the bootstrap
parameter estimate as well as of the bootstrap empirical cdf. In
contract, the bias term $C_n(x)$ is constructed using the empirical
cdf and by plugging in the sample estimator of the parameter (so it
does not involve bootstrap expectations).  This is explained in the
added section, Section 2.4.  This much is clear. However, what is
still not clear to me is WHY one should use the correction $K_n(x)$
instead of $C_n(x)$. The explanation in Section 2.4 is that the
bootstrap expectations become ``necessary only when serial dependence
distorts the properties of $F_n(x)$''.  But this statement is quite
vague.  Also vague is the statement in the preceding sentence which
says that the empirical cdf and the parameter estimate are ``reliable''
under independence -- suggesting that they are not ``reliable'' when
there is serial dependence in the data.  But what does ``reliable''
mean? These estimators should be consistent (under mild conditions) in
both settings.


So I am wondering: Does the performance of the NPBB method suffer when
the correction $C_n(x)$ is used instead of the correction $K_n(x)$?
Perhaps it is quite difficult to rigorously prove that it is necessary
or advantageous to use $K_n(x)$ instead of $C_n(x)$, but can you show
a difference in performance via simulation?  It may be that there is a
good reason to use $K_n(x)$ instead of $C_n(x)$, but the statements
given in the paper are not fully convincing.   It would suffice, in my
opinion, to run a simulation in which it is shown that using $C_n(x)$
instead of $K_n(x)$ results in worse performance.  Then one could
simply say that better performance is demonstrated when the correction
$K_n(x)$ is used (this would back up the heuristics given in added
section 2.4). Might it be that using $C_n(x)$ works out just as well
as using $K_n(x)$?  I think it is important to look into this, and I
don't believe it will take too much time. A simulation investigating
this could be placed in supplementary material.
\end{comment}


\textbf{Response to Comment on the Bias Correction Term.}

We thank the reviewer for raising this insightful question and for the
helpful suggestion to clarify the motivation for using $K_n(x)$ rather
than $C_n(x)$.  The previously confusing paragraph about the
``reliability'' of $F_n(x)$ and $\hat\theta_n$ has already been removed.
Section~2.4 now presents a streamlined discussion that first explains how
$K_n(x)$ centers the bootstrap KS process, then outlines the theoretical
justification for its use, and finally summarizes new simulation results.


A newly added paragraph clarifies that two bias corrections$C_n$ and
$K_n$ are asymptotically equivalent under independence and
short--memory dependence.  The preference for $K_n(x)$ arises from
theoretical considerations: under serial dependence, the bootstrap
analogue of $C_n(x)$ can exhibit a small $O_p(l/n)$ conditional bias
due to cross--lag covariances between $F_n(x)$ and $\hat\theta_n$,
where $l$ is the block length.  The expectation centering in $K_n(x)$
removes this bias without changing the first--order limit, yielding a
cleaner asymptotic argument.


To examine whether this refinement provides tangible benefits, we added
simulations comparing the two corrections.  Across all dependence levels
and sample sizes considered, the empirical sizes and powers were nearly
identical.  This outcome supports the theoretical result that both
methods converge to the same limit distribution and that the additional
expectation centering in $K_n(x)$ mainly improves conceptual symmetry
rather than finite--sample performance.  We have revised
Section~2.4 accordingly and added a brief reference in the Supplement to
document these findings.


\begin{comment}
The investigations into the performance of the NPBB under a
data-driven selection of the block size are appreciated.  If the paper
needs to be shortened, this part could be placed into a supplement,
and the results simply alluded to in the manuscript.
\end{comment}

Thank you for the suggestion, these investigations have now been
placed in the ``Selection of Block Size'' of the newly added
supplement, and they are still referred to in the main manuscript.

\begin{comment}
There is an odd typo on the bottom of page 9. It says ``average
empirical GXSdistribution function''
\end{comment}

Thank you for pointing this out. This typo has been
corrected to ``average empirical distribution function'' in the manuscript.



\bibliographystyle{chicago}
\bibliography{citations}


\end{document}
