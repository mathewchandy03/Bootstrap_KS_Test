\documentclass[12pt]{article}
\usepackage[margin=1in]{geometry}
\usepackage{natbib}
\usepackage{hyperref}

\usepackage{listings}
\usepackage{rotating, graphicx}
\graphicspath{{./}, {./image/}}
\usepackage{booktabs, natbib}
% \usepackage{breakurl}
% \usepackage [english]{babel}
\usepackage{amsmath, amsbsy, amsthm, epsfig, epsf, psfrag, graphicx, 
amssymb, enumerate}
\usepackage{bm}
\usepackage{multirow, multicol}

\usepackage[dvipsnames]{color}
\definecolor{darkblue}{rgb}{0.1, 0.2, 0.6}

\newcommand{\jy}[1]{\textcolor{red}{JY: #1}}
\newcommand{\eds}[1]{\textcolor{blue}{(EDS: #1)}}
\newcommand{\mc}[1]{\textcolor{green}{(MC: #1)}}
\newcommand{\xz}[1]{\textcolor{cyan}{(XZ: #1)}}

\sloppy

% \usepackage{csquotes}
% \usepackage [autostyle, english = american]{csquotes}
% \MakeOuterQuote{"}

% \usepackage{bibentry}
\newenvironment{comment}%
{\begin{quotation}\noindent\small\it\color{darkblue}\ignorespaces%
}{\end{quotation}}


\begin{document}

\begin{center}
  {\Large\bf Response to the Comments}
\end{center}


We thank the Editor and Associate Editor for the encouraging decision
and the opportunity to revise our manuscript. We are also grateful to
the two referees for their thoughtful and constructive comments. Their
suggestions have clarified important aspects of our work and have
helped sharpen the presentation. We appreciate the confidence shown in
our manuscript and have taken the revision process seriously in order
to meet the expectations for final acceptance.


The revised manuscript incorporates the following notable changes:
\begin{enumerate}
\item An appendix has been added documenting our exploration of
  automatic block size selection for circular block bootstrap,
  including observed limitations in practice.
\item  A paragraph has been added to the Discussion section to
  discuss the impact of block size on test performance and to
  contextualize our findings in light of existing literature on
  optimal block length
\item A stronger and more detailed justification of the bias
  correction term is now provided in a newly added Section~2.4.
\end{enumerate}


Point-by-point responses to the comments are as follows, with the
comments quoted in \emph{\color{darkblue} italic and blue}.


\subsection*{To AE}

\begin{comment}
Two reviewers have carefully read the paper and have provided detailed feedback.  Both reviewers would like to see some additional discussion and justifications for the bias correction term, where one could closely follow the \citet{babu2004goodness} paper, and one reviewer mentioned the importance of considering different block sizes (I recommend possibly trying some data-based block size selection method and checking how it performs).  A successful revision will require that the authors give serious consideration to these points.
\end{comment}


We thank the AE for the careful reading and encouraging feedback. The
reviewers’ comments have been addressed to the best of our ability in
the revised manuscript. We have added a heuristic justification for
the bias correction term in Section~2.4, following the logic of
\citet{babu2004goodness}, and clarified its extension to the block
bootstrap setting. As suggested, we also implemented a data-driven
block size selection method based on \citet{politis2004automatic} and
evaluated its performance. The results are included in the Appendix,
along with a discussion of the challenges posed by automatic block
length selection under strong serial dependence. Please find our
detailed point-by-point responses below.

\begin{comment}
1. I also noticed that on Page 3 it says “Employing a nonparametric method would necessitate defining a specific model for this dependence”, but I think here “nonparametric” should be “parametric”.
\end{comment}

Fixed.


\subsection*{To Referee 1}

\begin{comment}
This manuscript proposes a nonparametric block bootstrap (NPBB) method for applying the
Kolmogorov-Smirnov (KS) test to stationary time series data with unknown parameters and unspecified serial dependence. The authors claim that the method offers a robust nonparametric solution for evaluating the fit of marginal distributions when data are serially dependent, without
requiring assumptions about the form of this dependence. The paper is well-written and smooth
to read. However, upon review, I find that it lacks methodological and theoretical innovation.
\end{comment}


We thank the reviewer for the thoughtful comments and for recognizing
the clarity of the manuscript. While the methodological foundation
builds on existing work, the extension of the bias-corrected
nonparametric bootstrap of \citet{babu2004goodness} to stationary time
series with unspecified serial dependence is, to our knowledge, new
and practically valuable. As the reviewer noted, the paper targets a
general audience; accordingly, we added for a heuristic justification
of the bias correction term in Section~2.4 and clarified how the
approach adapts and differs from the i.i.d. setting in the presence of
serial dependence. We have revised the Introduction and Discussion
sections to better emphasize the scope and contribution.


\begin{comment}
1. The main contribution is presented as an extension of the bias-corrected 
nonparametric bootstrap (NPB) KS test, which was already established by 
\citet{babu2004goodness}
for independent data, to serially dependent time series. However, the 
methodological advancements
here are minimal; the approach essentially adapts existing block bootstrap 
methods and applies them without significant theoretical innovation or novel 
insights into the performance
of the KS statistic under these conditions. Notably, the numerical results 
suggest that the
proposed method struggles to control the size, which most likely stems from the 
uniform
choice of block size across different dependence structures. Intuitively, the 
block size should
vary with the level of dependence in the data; however, the manuscript lacks 
consideration
for this crucial aspect. I highly suggest the authors to include some 
discussions on this, see
some existing literature below.

\citet{hall1995blocking}

\citet{lahiri1999theoretical}

\citet{buhlmann2002bootstraps}

\citet{politis2004automatic}

\citet{lahiri2013resampling}

\end{comment}

We thank the reviewer for this thoughtful and constructive comment. 
We agree that block size selection is a critical component in the 
performance of block bootstrap procedures, particularly under varying 
levels of serial dependence. In response to this suggestion, we have 
made the following revisions:

\begin{itemize}
\item We have added a discussion to the conclusion section (Section~5) 
that explicitly addresses the role of block size and its effect on the 
performance of the NPBB test. This includes a review of relevant literature, 
as suggested by the reviewer, including works by 
\citet{hall1995blocking}, \citet{lahiri1999theoretical}, 
\citet{buhlmann2002bootstraps}, \citet{politis2004automatic}, and 
\citet{lahiri2013resampling}.

\item Motivated by these references, we implemented the automatic block 
length selection method proposed by \citet{politis2004automatic} for 
circular block bootstrap and reported our findings in a new appendix. 
While this method is attractive in principle, our empirical results show 
that it often selects overly large block sizes---especially under strong 
dependence---resulting in limited variability among bootstrap replicates 
and diminished test performance. These observations echo the theoretical 
concerns raised in the literature and underscore the practical limitations 
of fully automatic block length selection in the context of 
goodness-of-fit testing.
\end{itemize}

We hope these additions clarify our awareness of the issue and reflect 
our effort to better contextualize the finite-sample limitations of the 
NPBB method.



\begin{comment}
2. I understand that the paper’s intended scope may not require high 
technicality. However,
since the core methodology is based on bootstrap techniques, a more detailed 
heuristic
explanation of the procedure is essential, especially regarding the bias 
correction term.
\end{comment}


We appreciate the reviewer’s suggestion and agree that a more accessible, 
heuristic explanation enhances the clarity of the methodology. In response, 
we have added a new subsection (Section~2.4). It gives an intuitive
discussion of the role of the bias correction term $ K_n(x)$, explains
how it arises in the bootstrap world, and clarifies how it relates to
its counterpart $C_n(x)$ in the independent case. We hope this
addition helps readers gain a better understanding of the underlying
structure and motivation of the proposed method.


\begin{comment}
3. For time series data, there are many variants of bootstrap, such as circular 
block bootstrap,
stationary bootstrap, dependent wild bootstrap. I wonder how other methods 
perform in
your setting.
\end{comment} 

Our method is indeed based on the circular block bootstrap, a variant
of the moving block bootstrap originally proposed by
\citet{romano1992circular}, and this is now clearly stated throughout
the manuscript. We agree that investigating how other bootstrap
variants, such as the stationary bootstrap
\citep{politis1994stationary} and the dependent wild
bootstrap \citep{shao2010dependent}, perform in conjunction with our
bias correction scheme is a valuable direction for future work. In
response to this comment, we have added a discussion in the third
paragraph of the \emph{Discussion} Section, highlighting this
extension as a promising avenue for further investigation.


\subsection*{To Referee 2}

\begin{comment}
This paper focuses on approximating the distribution of the Kolmogorov-
Smirnov (KS) statistic for stationary time series. Specifically, it deals with the
distribution of KS statistic for testing if the marginal distribution belongs to
a specified parametric family. Parametric bootstrap requires knowledge of the
nature of serial dependence, and for this reason, nonparametric bootstrap is 
employed here. For the iid case, \citet{babu2004goodness} obtained a correction 
term so that the corrected bootstraped statistic can be used to approximate 
distribution of the KS statistic. This paper does the same except it uses 
nonparametric bootstrap developed by Kunsch and others. Finally the author(s) 
present some empirical work in support of their proposed procedure. Simulations 
indicate that the proposed procedure may work for large samples and
moderate dependence.
\end{comment}

We thank the reviewer for their accurate and concise summary of our
contribution.

\begin{comment}
1. In Section 3.1, it seems that Kendall's tau is being used as a measure
dependence between $X_t$ and $X_{t+1}$. It may be useful to make that explicit 
in the second paragraph of Section 3.1.
\end{comment}

Thank you for pointing this out. In the second paragraph of
Section~3.1, we have added that the serial dependence is specifically
between $X_t$ and $X_{t+1}$.

\begin{comment}
2. The work presented here is useful, however I have believe the explanations
for the correction term may need modification. Here the entire justification is
based on the null case which does not cover the case when the null is not true.
Justification for the correction term in \citet{babu2004goodness} is not based 
on just the
null case. The same seems to be true here, and in that case, Section 2 needs
rewriting.

Here is what I am able to understand after reading this paper, the papers
by \citet{babu2004goodness}, and \citet{kunsch1989jackknife} and others.

Following the notations in this paper (the same notations have been used in
\citet{babu2004goodness}), let

\begin{align*}
Y_n(x, \hat \theta_n) &= \sqrt{n}[F_n(x) - F(x; \hat \theta_n)],\\
Y_n^{(b)}(x, \hat \theta_n^{(b)}) &= \sqrt{n}[F_n^{(b)}(x) - F(x; \hat \theta_n^{(b)})],
\end{align*}

where the second quantity is based on the $b$th bootstrap sample. In the iid 
case, \citet{babu2004goodness} introduced a correction term for the nonparametric bootstrap
$$B_n(x) = \sqrt{n}[F_n(x) - F(x; \hat \theta_n)],$$
so that
\begin{align*}
Y_{n,corr}^{(b)}(x) &= Y_n^{(b)}(x; \hat\theta_n^{(b)}) - B_n(x)\\
                    &= \sqrt{n}[F_n^{(b)}(x) - F_n(x)] -
                    \sqrt{n}[F(x; \hat \theta_n^{(b)}) -  F(x; \hat\theta_n)].
\end{align*}
It was proved in \citet{babu2004goodness} that the distribution of the KS
statistic is approximated by that of the corrected bootstrap under the null as
well as for contiguous alternatives.

The work under consideration uses nonparametric block bootstrap. Let 
$\theta^*_n = E^*[\hat \theta_n ^{(b)}]$ and $F^*_n(x) = E^*[F_n^{(b)}(x)]$,
where $E^*$ represents average over bootstrap samples. In this paper the 
correction term is
$$C_n(x) = \sqrt{n}[F^*_n(x)-F(x;\theta^*_n)],$$
so that
\begin{align*}
Y^{(b)}_{n,corr}(x) &= Y^{(b)}_n(x;\hat \theta_n^{(b)})-C_n(x)\\
                    &= \sqrt{n}[F_n^{(b)}(x) - F^*_n(x)] 
                    - \sqrt{n}[F(x; \hat\theta_n^{(b)}) - F(x; \theta^*_n)].
\end{align*}
So the corrected bootstrap proposed here has the same form as in Babu and Rao,
and the justifications for the procedure proposed here should also be similar.
\end{comment}

Thank you for drawing this to our attention. As is standard in hypothesis 
testing, our development of the bootstrap procedure is focused on accurately 
approximating the distribution of the test statistic under the null hypothesis. 
This is consistent with the goal of using the bootstrap to construct valid
 p-values and rejection regions. While the same bias correction is applied 
regardless of whether the null is true, its purpose is to replicate the 
behavior of the statistic under the null, which is the basis for valid inference.


To better explain the rationale for the correction term, and in response to both 
this comment and Reviewer 1’s request for clearer heuristic explanation, we have 
added a dedicated subsection (Section~2.4) titled \emph{Heuristic justification 
of the bias correction in NPBB}. This subsection provides an intuitive discussion 
of the role of the bias correction term, its construction, and its relation to 
the version used in the independent case. The revised text aims to clarify that 
the correction is justified by its ability to improve bootstrap calibration under 
the null hypothesis, which is the relevant distribution for hypothesis testing.


\bibliographystyle{chicago}
\bibliography{citations}


\end{document}
