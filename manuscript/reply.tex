\documentclass[12pt]{article}
\usepackage[margin=1in]{geometry}
\usepackage{natbib}
\usepackage{hyperref}

\usepackage{listings}
\usepackage{rotating, graphicx}
\graphicspath{{./}, {./image/}}
\usepackage{booktabs, natbib}
% \usepackage{breakurl}
% \usepackage [english]{babel}
\usepackage{amsmath, amsbsy, amsthm, epsfig, epsf, psfrag, graphicx, 
amssymb, enumerate}
\usepackage{bm}
\usepackage{multirow, multicol}

\usepackage[dvipsnames]{color}
\definecolor{darkblue}{rgb}{0.1, 0.2, 0.6}

\newcommand{\jy}[1]{\textcolor{red}{JY: #1}}
\newcommand{\eds}[1]{\textcolor{blue}{(EDS: #1)}}
\newcommand{\mc}[1]{\textcolor{green}{(MC: #1)}}
\newcommand{\xz}[1]{\textcolor{cyan}{(XZ: #1)}}

\sloppy

% \usepackage{csquotes}
% \usepackage [autostyle, english = american]{csquotes}
% \MakeOuterQuote{"}

% \usepackage{bibentry}
\newenvironment{comment}%
{\begin{quotation}\noindent\small\it\color{darkblue}\ignorespaces%
}{\end{quotation}}


\begin{document}

\begin{center}
  {\Large\bf Response to the Comments}
\end{center}

We extend our gratitude to the editor and the Associate Editor for
granting us the opportunity to revise this manuscript. We also want to
express our appreciation to the two referees for their valuable and
constructive comments. The reviewers have highlighted crucial
discussion areas that have not only enhanced the quality of this paper
but have also set a clear direction for our future research
endeavors.


The manuscript has been
revised accordingly with the following notable changes:
\begin{enumerate}
\item 
\item 
\item 
\item 
\end{enumerate}


Point-by-point responses to the comments are as follows, with the
comments quoted in \emph{\color{darkblue} italic and blue}.

\subsection*{To Referee 1}

\begin{comment}
1. The main contribution is presented as an extension of the bias-corrected 
nonparametric
bootstrap (NPB) KS test, which was already established by Babu and Rao (2004) 
for independent data, to serially dependent time series. However, the 
methodological advancements
here are minimal; the approach essentially adapts existing block bootstrap 
methods and applies them without significant theoretical innovation or novel 
insights into the performance
of the KS statistic under these conditions. Notably, the numerical results 
suggest that the
proposed method struggles to control the size, which most likely stems from the 
uniform
choice of block size across different dependence structures. Intuitively, the 
block size should
vary with the level of dependence in the data; however, the manuscript lacks 
consideration
for this crucial aspect. I highly suggest the authors to include some 
discussions on this, see
some existing literature below.

\citep{hall1995blocking}

\citep{lahiri1999theoretical}

\citep{buhlmann2002bootstraps}

\citep{politis2004automatic}

\citep{lahiri2013resampling}

\end{comment}

Thank you for your suggestion. The primary methodological 
advancement in this study is a new bias correction - which is different than 
that introduced by \citet{babu2004goodness} - for cases where the data are
serially dependent...



\begin{comment}
2. I understand that the paper’s intended scope may not require high 
technicality. However,
since the core methodology is based on bootstrap techniques, a more detailed 
heuristic
explanation of the procedure is essential, especially regarding the bias 
correction term.
\end{comment}

Thank you for your comment...



\begin{comment}
3. For time series data, there are many variants of bootstrap, such as circular 
block bootstrap,
stationary bootstrap, dependent wild bootstrap. I wonder how other methods 
perform in
your setting.
\end{comment} 

This is an interesting direction to investigate...



\subsection*{To Referee 2}

\begin{comment}
1. In Section 3.1, it seems that Kendall's tau is being used as a measure
dependence between $X_t$ and $X_{t+1}$. It may be useful to make that explicit 
in the
second pararaph of Section 3.1.
\end{comment}

Thank you for pointing this out, we have added that the
serial dependence is specifically between $X_i$ and $X_{i+1}$.




\begin{comment}
2. The work presented here is useful, however I have believe the explanations
for the correction term may need modification. Here the entire justification is
based on the null case which does not cover the case when the null is not true.
Justification for the correction term in Babu and Rao is not based on just the
null case. The same seems to be true here, and in that case, Section 2 needs
rewriting.
\end{comment}

Thank you for drawing this to our attention...





\bibliographystyle{chicago}
\bibliography{citations}


\end{document}
